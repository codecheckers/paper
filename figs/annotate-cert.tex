\documentclass{article}
\usepackage[svgnames]{xcolor}
\usepackage{tikz,pdfpages}
\usepackage{mathpazo}


%% taken from https://tex.stackexchange.com/questions/85651/is-there-are-way-to-annotate-pdfs-with-latex
\usepackage{tikz}

\tikzset{
  every node/.style={
    anchor=mid west,
  }
}

\makeatletter
\pgfkeys{/form field/.code 2 args={\expandafter\global\expandafter\def\csname field@#1\expandafter\endcsname\expandafter{#2}}}

\newcommand{\place}[3][]{\node[#1] at (#2) {\csname field@#3\endcsname};}
\makeatother

\newcommand{\xmark}[1]{\node at (#1) {X};}


\begin{document}


\includepdf[pages=1,clip,trim=0.5cm 3cm 0.5cm 3cm,
  picturecommand={%
    \begin{tikzpicture}[remember picture,overlay]
%%% The next lines draw a useful grid - get rid of them (comment them out) on the final version
%     \draw[gray] (current page.south west) grid (current page.north east);
% \foreach \k in {1,...,28} {
%       \path (current page.south east) ++(-2,\k) node {\k};
% }
% \foreach \k in {1,...,20} {
%       \path (current page.south west) ++(\k,2) node {\k};
% }
%%% grid code ends here
\tikzset{every node/.append style={fill=White,font=\Large}}
%% \draw (2,25) circle [radius=0.3] node {(A)};
\draw (0.15,26.2) node {A};
\draw[rounded corners,draw=cyan] (0.20,24) rectangle ++(10,2.7);
\draw (0.15,23) node {B};
\draw[rounded corners,draw=cyan] (0.20,20.0) rectangle ++(10,3.5);
\draw (0.15,19.3) node {C};
\draw[rounded corners,draw=cyan] (0.20,16.0) rectangle ++(10,3.8);
\draw (11,23.5) node {D};
\draw[rounded corners,draw=cyan] (11,18.5) rectangle ++(10,5.5);
\draw (0.15,12.5) node {E};
\draw[rounded corners,draw=cyan] (0.20,6.5) rectangle ++(10,6.6);
\draw (11,12.5) node {F};
\draw[rounded corners,draw=cyan] (11,4) rectangle ++(10,9.0);
\draw[rounded corners,draw=cyan] (0,14) rectangle ++(10.7,13.0);
\draw[rounded corners,draw=cyan] (10.8,14) rectangle ++(10.7,13.0);
\draw[rounded corners,draw=cyan] (0,0.9) rectangle ++(10.7,13.0);
\draw[rounded corners,draw=cyan] (10.8,0.9) rectangle ++(10.7,13.0);
\draw (5,15) node {Page 1};
\draw (16,15) node {Page 2};
\draw (5,2) node {Page 3};
\draw (16,2) node {Page 4};
    \end{tikzpicture}
  }
]{cert-2020-012-4up.pdf}
\end{document}
